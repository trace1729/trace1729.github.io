% Do not alter this block (unless you're familiar with LaTeX
\documentclass[]{article}
\usepackage[utf8]{inputenc}
\usepackage{ctex}
\usepackage[margin=1in]{geometry} 
\usepackage{amsmath,amsthm,amssymb,amsfonts, fancyhdr, color, comment, graphicx, environ}
\usepackage{xcolor}
\usepackage{mdframed}
\usepackage[shortlabels]{enumitem}
\usepackage{indentfirst}
\usepackage{hyperref}
\usepackage{fontenc}
\usepackage{xeCJK}
%\newcommand{\lxgw}{\CJKfontspec{LXGWWenKaiScreen.ttf}}%华文行楷
\setCJKmainfont{LXGWWenKaiScreen.ttf}
\hypersetup{
    colorlinks=true,
    linkcolor=blue,
    filecolor=magenta,      
    urlcolor=blue,
}


\pagestyle{fancy}


\newenvironment{problem}[2][Problem]
    { \begin{mdframed}[backgroundcolor=gray!20] \textbf{#1 #2} \\}
    {  \end{mdframed}}

% Define solution environment
\newenvironment{solution}
    {\textit{Proof:}}
    {\\ \\ \\ \\ \\ \\ \\ \\ \\}

\renewcommand{\qed}{\quad\qedsymbol}

% prevent line break in inline mode
\binoppenalty=\maxdimen
\relpenalty=\maxdimen

%%%%%%%%%%%%%%%%%%%%%%%%%%%%%%%%%%%%%%%%%%%%%
%Fill in the appropriate information below
\lhead{Your name: }
\rhead{CSCI 2824-310} 
\chead{\textbf{第十五 十四届选择题}}
%%%%%%%%%%%%%%%%%%%%%%%%%%%%%%%%%%%%%%%%%%%%%
\begin{document}

\begin{problem}{1}
     已知可导函数$ f(x) $满足 $ \displaystyle{f(x)}
     \cos x + 2\int_{0}^{x} f(t) \sin t \, \, \mathrm{d}t = x + 1  $
     则$ f(x) = ? $
\end{problem}
\begin{solution}
\end{solution}

\begin{problem}{2}
    \begin{displaymath}
        \lim_{n \to \infty} \sin ^2(\pi \sqrt{n^2 + 1}) = 
    \end{displaymath}
\end{problem}
\begin{solution}
\end{solution}


\begin{problem}{3}
     \begin{displaymath}
        \iint \limits_{|x| + |y| \le 1} (|x| + |y|) \mathrm{d}x \mathrm{d}y =
     \end{displaymath}
\end{problem}
\begin{solution}
\end{solution}

\begin{problem}{4}
     \begin{displaymath}
        \lim_{n \to \infty} \frac{k}{n^2 + k^2} = 
     \end{displaymath}
\end{problem}
\begin{solution}
\end{solution}

\begin{problem}{5} 
     曲面$ z = x^2 + y^2 $ 在 $ (1, -1, 3) $的切平面方程为
\end{problem}
\begin{solution}
\end{solution}


\begin{problem}{6}
     \begin{displaymath}
        \int_{0}^{\frac{\pi}{2}} \frac{\mathrm{d}x}{1 + \tan ^4 x}
     \end{displaymath}
\end{problem}
\begin{solution}
\end{solution}

\begin{problem}{7}
     微分方程$\frac{\mathrm{d} y}{\mathrm{d} x} -xy  = xe^{x^2}$ 
     满足 $ y(0) = 1 $ 的特解为 
\end{problem}
\begin{solution}
\end{solution}

\begin{problem}{8}
     设$ D: 1 \le x^2 + y^2 \le 4 $, 则 $ \displaystyle{\iint \limits_{D} (x + y^2)e^{-(x^2 + y^2 - 4) 
    } \mathrm{d}x \mathrm{d}y }$
\end{problem}
\begin{solution}
\end{solution}

\begin{problem}{9}
     \begin{displaymath}
        \lim_{n \to \infty} \Biggl( \frac{1}{n + 1} + \cdots + \frac{1}{n+n} \Biggr)
     \end{displaymath}
\end{problem}
\begin{solution}
\end{solution}

\begin{problem}{10}
     \begin{displaymath}
        \lim_{x \to 0} \biggl( \frac{1}{x^2} - \cot ^2 x \biggr)
     \end{displaymath}
\end{problem}
\begin{solution}
\end{solution}

\begin{problem}{11}
     设 $f(t)$ 二阶连续可导,且$ f(t) \neq 0 $, 若 $\displaystyle{ 
        x = \int_{0}^{t} f(s) \, \, \mathrm{d}s, y = f(t) 
      }$,则 $ \displaystyle{ \frac{\mathrm{d}^2 y}{\mathrm{d} x^2} =} $
\end{problem}
\begin{solution}
\end{solution}



\begin{problem}{Reflection}
\textbf{[3pts]} Exercise a growth mentality by reflecting on this assignment and your work. Feel free to say whatever you want, but you are required to answer the following. You are graded on whether you complete this, not on what you say.
\begin{itemize}
\item How many hours did you spend on this assignment?
\item What problem was hardest? Why?
\item What problem was easiest? Why?
\end{itemize}
\end{problem}

\end{document}